%%
%% This is file `sample-sigplan.tex',
%% generated with the docstrip utility.
%%
%% The original source files were:
%%
%% samples.dtx  (with options: `sigplan')
%% 
%% IMPORTANT NOTICE:
%% 
%% For the copyright see the source file.
%% 
%% Any modified versions of this file must be renamed
%% with new filenames distinct from sample-sigplan.tex.
%% 
%% For distribution of the original source see the terms
%% for copying and modification in the file samples.dtx.
%% 
%% This generated file may be distributed as long as the
%% original source files, as listed above, are part of the
%% same distribution. (The sources need not necessarily be
%% in the same archive or directory.)
%%
%% Commands for TeXCount
%TC:macro \cite [option:text,text]
%TC:macro \citep [option:text,text]
%TC:macro \citet [option:text,text]pages
%TC:envir table 0 1
%TC:envir table* 0 1
%TC:envir tabular [ignore] word
%TC:envir displaymath 0 word
%TC:envir math 0 word
%TC:envir comment 0 0
%%
%%
%% The first command in your LaTeX source must be the \documentclass command.
\documentclass[sigplan,screen]{acmart}
%% NOTE that a single column version is required for 
%% submission and peer review. This can be done by changing
%% the \doucmentclass[...]{acmart} in this template to 
%% \documentclass[manuscript,screen,review]{acmart}
%% 
%% To ensure 100% compatibility, please check the white list of
%% approved LaTeX packages to be used with the Master Article Template at
%% https://www.acm.org/publications/taps/whitelist-of-latex-packages 
%% before creating your document. The white list page provides 
%% information on how to submit additional LaTeX packages for 
%% review and adoption.
%% Fonts used in the template cannot be substituted; margin 
%% adjustments are not allowed.
%%
%% \BibTeX command to typeset BibTeX logo in the docs
\AtBeginDocument{%
  \providecommand\BibTeX{{%
    \normalfont B\kern-0.5em{\scshape i\kern-0.25em b}\kern-0.8em\TeX}}}

%% Rights management information.  This information is sent to you
%% when you complete the rights form.  These commands have SAMPLE
%% values in them; it is your responsibility as an author to replace
%% the commands and values with those provided to you when you
%% complete the rights form.
\setcopyright{acmlicensed}
\copyrightyear{2018}
\acmYear{2018}
\acmDOI{XXXXXXX.XXXXXXX}

%
%  Uncomment \acmBooktitle if th title of the proceedings is different
%  from ``Proceedings of ...''!
%
%\acmBooktitle{Woodstock '18: ACM Symposium on Neural Gaze Detection,
%  June 03--05, 2018, Woodstock, NY} 
\acmISBN{978-1-4503-XXXX-X/18/06}


%%
%% Submission ID.
%% Use this when submitting an article to a sponsored event. You'll
%% receive a unique submission ID from the organizers
%% of the event, and this ID should be used as the parameter to this command.
%%\acmSubmissionID{123-A56-BU3}

%%
%% For managing citations, it is recommended to use bibliography
%% files in BibTeX format.
%%
%% You can then either use BibTeX with the ACM-Reference-Format style,
%% or BibLaTeX with the acmnumeric or acmauthoryear sytles, that include
%% support for advanced citation of software artefact from the
%% biblatex-software package, also separately available on CTAN.
%%
%% Look at the sample-*-biblatex.tex files for templates showcasing
%% the biblatex styles.
%%

%%
%% The majority of ACM publications use numbered citations and
%% references.  The command \citestyle{authoryear} switches to the
%% "author year" style.
%%
%% If you are preparing content for an event
%% sponsored by ACM SIGGRAPH, you must use the "author year" style of
%% citations and references.
%% Uncommenting
%% the next command will enable that style.
%%\citestyle{acmauthoryear}

%%
%% end of the preamble, start of the body of the document source.
\begin{document}

%%
%% The "title" command has an optional parameter,
%% allowing the author to define a "short title" to be used in page headers.
\title{DDoS defence mechanisims }

%%
%% The "author" command and its associated commands are used to define
%% the authors and their affiliations.
%% Of note is the shared affiliation of the first two authors, and the
%% "authornote" and "authornotemark" commands
%% used to denote shared contribution to the research.
\author{Alexander Haar}
\email{Haaralexander3@gmail.com}
\affiliation{%
  \institution{Universität Kassel}
  \city{Kassel}
  \state{Hessen}
  \country{Deutschland}
}

%%
%% The abstract is a short summary of the work to be presented in the
%% article.
\begin{abstract}
  Abstract TBD
\end{abstract}

%%
%% Keywords. The author(s) should pick words that accurately describe
%% the work being presented. Separate the keywords with commas.
\keywords{Source Address Validation, IP Spoofing, IXP, Scrubber, Machine Learning}


%%
%% This command processes the author and affiliation and title
%% information and builds the first part of the formatted document.
\maketitle

\section{Introduction}
Cyberkriminalität ist in der Modernen Gesellschaft nicht mehr wegzudenken. Insbesondere die DDoS Attacken gegen Unternehmen und staatliche Institutionen haben in den letzten Jahren dramatisch zugenommen.

\section{IXP Scrubber}
Zuerst wird erklärt was ein Internet Exchange Point (IXP) ist und wo er sich im Internet befindet. Danach erfolgt eine Einführung in den IXP Scrubber. Wie er funktioniert und wie er schafft schädlichen Internetverkehr zu filtern.

\subsection{Internet Exchange Point}
Das Internet bestehe aus einer Vielzahl von miteinander verbundenen Netzwerken. Um diese Netzwerke effizient zu verknüpfen, werden sogenannte Internet Exchange Points (IXPs) am Rand der Netzwerke platziert \cite{Cloudflare01}. Diese IXPs fungieren als zentrale Knotenpunkte im Internet und ermöglichen es, den Internetverkehr von einem Netzwerk auf ein anderes zu leiten. Praktisch betrachtet handelt es sich bei IXPs um Schaltstellen, die als Bindeglieder zwischen verschiedenen Netzwerken dienen.

Die Funktionsweise der IXPs kann mit der eines Internet-Switches verglichen werden, wobei sie auf der zweiten Ebene des OSI-Modells positioniert sind. Zur Kommunikation mit einem IXP verwenden Netzwerke das sogenannte Backbone-Protokoll.

Ein wesentlicher Vorteil von IXPs liegt in der geringen Latenz und der kurzen Roundtrip-Zeit. Darüber hinaus ermöglichen sie es, den Verkehr im Falle eines Ausfalls auf alternative Routen umzuleiten, was die Robustheit des Netzwerks erhöht. Dies trägt dazu bei, dass das Netzwerk weniger anfällig für Fehler und Störungen ist. \cite{Cloudflare01}

\subsection{Scrubber}
\cite{Hohlfeld01}

%%
%% The next two lines define the bibliography style to be used, and
%% the bibliography file.
\bibliographystyle{ACM-Reference-Format}
\bibliography{sample-base}

%%
%% If your work has an appendix, this is the place to put it.
\appendix

\section{Research Methods}

\subsection{Part One}

TBD


\section{Online Resources}

TBD

\end{document}
\endinput
%%
%% End of file `sample-sigplan.tex'.
